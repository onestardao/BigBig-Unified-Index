% LaTeX preamble and document content
\documentclass[11pt]{article}
\usepackage{amsmath,amssymb,url,hyperref,listings}
\usepackage[margin=1in]{geometry}
\usepackage{graphicx}
\usepackage{tikz}                % 只是讓編譯器有 tikz(即使不畫圖)
\usepackage{color}
\newcommand{\tc}{\mathrm{TC}}   
\newcommand{\SM}{\mathrm{SM}}   


\title{
  \textbf{Beyond the Higgs:A WhiteCrow (HPC-Driven) Survey of Hybrid Mass Generation}\\
\\
  \vspace{0.3em}
  \large (v4.4 Beta)
}

\author{
  \textbf{PSBigBig}\\
  \textit{Independent Developer}\\
  \href{mailto:hello@onestardao.com}{hello@onestardao.com}\\
  \url{https://onestardao.com}\\
  \url{https://linktr.ee/onestardao}
}

\date{2025/5/16}

\begin{document}


\maketitle

\section*{Beta Notice (Work in Progress)}
\label{sec:beta-notice}
This document is released as a Beta version (v4.4 Beta) under continuous development. 
No final peer review or official acceptance is claimed at this stage. 
Readers and reviewers are encouraged to further test or verify the methodology, 
particularly regarding HPC meltdown illusions and the WhiteCrow scenario detection.

\section*{Short Unified Disclaimer}
This study presents a conceptual and computational pipeline involving 
High-Performance Computing (HPC) meltdown illusions and WhiteCrow flipping phenomena. 
All statements remain open to future confirmation and are not asserted to be final. 
Use of any derived code or data is at the researcher's own discretion.

% ---------------------
\begin{abstract}
The emergence of HPC meltdown illusions 
in various Beyond-Standard-Model (BSM) explorations 
has raised questions about whether subtle numerical instabilities 
could obscure or mimic new physics signals. 
In this work, we introduce a multi-round AI-driven checking pipeline, 
capable of dynamically adjusting parameters in a Toy HPC model 
to capture ``WhiteCrow'' scenarios, where small parameter shifts 
cause dramatic flips in predicted exotic signatures. 
We demonstrate that even a modest meltdown factor can enable 
mass predictions to swing from near-0~TeV to above 3~TeV, 
thus flipping the exotic-flag from 0 to 2 (or vice versa). 
We further illustrate a procedure to iteratively detect and confirm 
over 100 such WhiteCrow events, emphasizing a ``not being hasty'' principle 
that employs multi-round (10\% then 5\%) parameter shifts 
for robust detection. 
Our findings suggest meltdown illusions might be far more prevalent 
in HPC-based BSM analyses than previously assumed, 
highlighting the importance of a carefully designed pipeline 
when searching for new physics signals or validating their absence.
\end{abstract}

% ---------------------
\section{Introduction}
\label{sec:intro}

\subsection{Background and Motivation}
Recent progress in High-Performance Computing (HPC) has facilitated 
the exploration of strongly coupled theories and high-dimensional parameter spaces 
in particle physics. However, the phenomenon often referred to as 
``meltdown illusions'' has repeatedly shown that HPC-based predictions 
can be unusually sensitive to seemingly minor shifts in lattice size, 
algorithmic precision, or boundary conditions. Such illusions 
may lead to seemingly contradictory results or abrupt flips 
in the predicted existence of exotic resonances.

Simultaneously, theoretical frameworks like \textit{Technicolor}, 
\textit{Composite Higgs}, or \textit{Extra Dimensions} 
have attempted to extend the Standard Model (SM) 
to explain phenomena including dark matter mass, neutrino properties, 
and potential new resonances at multi-TeV scales. 
Yet these theories often rely heavily on HPC numerical simulations 
to validate or exclude certain mass ranges or couplings.

In parallel, the \textbf{BigBig Unity Formula (Beta)} has been proposed 
as a conceptual approach to unify various BSM attempts while acknowledging 
the incomplete nature of each. This Beta version remains a work in progress, 
but it provides a flexible vantage to investigate how HPC meltdown illusions 
might hamper or artificially promote the detection of new physics signals.

\subsection{WhiteCrow Scenario and Flipping Phenomena}
Within this context, a ``WhiteCrow scenario'' is defined as a case 
where minimal parameter shifts---for instance, $\pm 10\%$ in an HPC input--- 
cause major flips in predicted exotic behavior (from ``no exotic'' to 
``strong exotic'' or vice versa). If HPC meltdown illusions are present, 
these WhiteCrow scenarios become significantly more frequent, 
meaning standard HPC-based analyses might prematurely exclude or confirm 
BSM signals depending on minor numerical variations.

In simpler terms, encountering even a single WhiteCrow scenario can refute 
the notion that HPC predictions are fully stable. However, identifying 
a single outlier might be attributed to a rare fluctuation or numerical noise. 
Hence, capturing a larger number of such events (e.g., 100 WhiteCrow flips) 
demonstrates a more systematic meltdown illusions effect. 
This is precisely the objective of our iterative pipeline: 
\textit{not being hasty}, but rather scanning extensively while performing 
multi-round parameter checks.

\subsection{Scope and Outline}
This paper focuses on a Toy HPC model that encodes meltdown illusions 
through a simple meltdown factor triggered by small lattice sizes 
or partial boundary mismatch. We then apply an AI-based multi-round 
checking mechanism, scanning the parameter space $(\alpha_{\mathrm{TC}}, N_f, L_{\mathrm{size}})$ 
for WhiteCrow flips. 

While we do not claim an immediate tie to real collider data at the Large Hadron Collider (LHC), 
the methodology and cautionary lessons drawn here are relevant 
for any HPC-based BSM search, and could be extended to real HPC codes 
or advanced lattice gauge computations in future work.

In Section~\ref{sec:background}, we review meltdown illusions and existing BSM HPC studies. 
Then, Section~\ref{sec:methodology} details our proposed AI pipeline, 
including the multi-round parameter shift and flipping detection. 
Section~6 (Implementation) explains how we run iteration logs, 
while Section~7 (Results) showcases capturing up to 100 WhiteCrow flips. 
Finally, we discuss implications and conclude with future outlook.

\subsection{Beta Status}
As indicated in the Beta Notice, the entire BigBig Unity Formula approach 
remains an evolving concept. The v4.4 Beta label underlines 
that certain definitions (e.g., meltdown illusions threshold) 
or WhiteCrow flipping criteria (base-flag = 1 vs.\ $\pm 10\%$ => 0,2) 
may still be refined in subsequent releases. 
Nevertheless, the pipeline already demonstrates 
the substantial presence of meltdown illusions 
in HPC-based BSM parameter scans, a phenomenon we believe warrants urgent attention.

\noindent\rule{\textwidth}{0.5pt}

\section{Theoretical Background and Literature Review}
\label{sec:background}

\subsection{Meltdown Illusions in HPC-Driven Theories}

High-Performance Computing (HPC) has played a decisive role in modern physics,
especially when exploring strongly coupled theories such as Technicolor, 
Composite Higgs, or even the interplay of extra-dimensional frameworks. 
However, practitioners have noted that numerical instabilities, 
which we label as \textit{meltdown illusions}, often surface whenever 
the underlying lattice or discretization scheme is undersized, 
or the boundary conditions are only partially consistent. 
In such a situation, even a small shift in lattice size $L_{\mathrm{size}}$ 
or coupling parameters $(\alpha_{\mathrm{TC}}$, $N_f)$
might yield disproportionately large deviations in mass predictions 
or exotic resonance signals.

Several early reports \cite{atlas, meltdownRef} 
suggest that meltdown illusions can appear if the HPC algorithm 
applies approximate methods for gauge coupling integrations 
while also making certain simplifying assumptions about 
background fields or iterative solvers. One distinctive trait is 
the abrupt jump in the predicted mass scale, which can range 
from an apparently negligible sub-TeV region to a multi-TeV domain. 
This abruptness is precisely what the WhiteCrow scenario in our pipeline 
seeks to detect: a mass flip from near zero to well above 3~TeV 
(or vice versa) after small parameter adjustments.

\subsection{Existing Studies on BSM Theories and Numerical Instabilities}

Numerous BSM frameworks---including walking Technicolor, 
Randall-Sundrum extra-dimensional models, and various composite scenarios---%
rely on HPC-based lattice simulations to pin down 
particular mass spectra or potential new resonances. 
In the simplest sense, such HPC calculations approximate 
continuous gauge theories with discrete lattices, 
attempting to maintain gauge invariance and 
renormalization consistency. Yet the history of lattice QCD 
has taught us that minor changes in boundary conditions 
or hopping parameters can lead to unexpectedly large shifts 
in the extracted hadron masses or decay constants \cite{qcdLatticeRef}. 
By extension, BSM lattice computations are arguably more fragile 
owing to less established priors or known constraints.

In high-dimensional parameter spaces, 
where $(\alpha_{\mathrm{TC}}, N_f, L_{\mathrm{size}})$ and possibly 
additional couplings or $N_f$ expansions come into play, 
the meltdown illusions phenomenon becomes even more pronounced. 
Low or moderate lattice sizes (e.g., $L_{\mathrm{size}} < 60$) 
are often used in feasibility tests to reduce computational cost. 
But this practice naturally fosters meltdown illusions 
because the algorithm has to extrapolate or guess 
the continuum limit from insufficient discretization data. 
Hence, preliminary HPC scans might erroneously discard 
or champion certain BSM models purely due to illusions 
masked as legitimate signals.

\subsection{WhiteCrow Scenario: Flips from 0 to 2 or 2 to 0}

The notion of flipping an \textit{exotic\_flag} from 0 to 2 
(or 2 to 0) upon a mild parameter shift is central 
to what we call the \textbf{WhiteCrow scenario}. 
Historically, many HPC-based BSM analyses 
focused on whether a predicted heavy resonance 
exceeds some threshold, e.g., 2~TeV or 3~TeV, 
leading to a yes/no classification for new physics feasibility. 
Nevertheless, meltdown illusions imply 
that an HPC code might predict mass $\approx 1.9$~TeV under one set of inputs 
(base\_flag $= 0$ indicating no strong exotic), 
then shift to $\approx 3.2$~TeV under a $+10\%$ change in $\alpha_{\mathrm{TC}}$, 
thus flipping to base\_flag $= 2$ (significant exotic).

Prior references \cite{whitecrowInitialRef, meltdownRef} 
hint at the existence of such abrupt flips 
but did not formulate an explicit pipeline to \textit{capture} 
multiple flipping events in a large-scale parameter exploration. 
Thus, WhiteCrow scenario detection had typically been anecdotal: 
authors would note a suspicious jump 
and treat it as an anomaly or numerical glitch. 
In contrast, we systematically incorporate 
multi-round checks to confirm that such flips persist 
even after a second or third mild parameter shift (e.g., $\pm 5\%$). 
Consequently, we can rule out ephemeral noise 
and declare a robust WhiteCrow event.

\subsection{Previous HPC Strategies and Gaps}

Most HPC-based BSM computations revolve around 
either scanning broad parameter regions with coarse stepping 
or running deeper, fine-lattice simulations at a small number of points. 
In either approach, meltdown illusions might remain hidden 
if no intermediate micro-shift is tested. 
Furthermore, meltdown illusions might mislead 
collider-based phenomenologists 
who interpret HPC results at face value. 
When the HPC output claims an exotic mass at 1.8~TeV 
(where LHC data might exclude or remain inconclusive), 
one might unknowingly disregard 
the possibility that a small $+10\%$ parameter shift 
pushes the same predicted mass to 3.1~TeV, 
potentially reviving the BSM candidate.

Our approach diverges from standard HPC scanning 
by introducing an AI-driven multi-round pipeline 
that systematically checks $\pm (10\%, 5\%)$ changes, 
detects meltdown illusions if mass\_diff $> 10\%$, 
and flags the presence of WhiteCrow scenario 
when flipping from a non-exotic domain to an exotic domain 
(or vice versa). A more thorough discussion 
of the methodology follows in Section~\ref{sec:methodology}, 
while the actual implementation and iteration logs 
are presented in later sections of this paper.

\subsection{Synopsis of This Work}

In summary, meltdown illusions have periodically emerged 
in HPC BSM analyses but rarely received methodical 
and repeated scanning attention. We propose 
a structured pipeline that identifies large flipping events, 
labels them as WhiteCrow (WC) scenarios, 
and accumulates the count of WC events 
until it reaches a significant threshold (e.g., 100). 
Such a threshold-based approach offers a statistical demonstration 
that meltdown illusions are not merely a one-off oddity, 
but potentially widespread in HPC-based new physics exploration. 
Moreover, the AI-driven multi-round approach ensures 
we remain ``not being hasty'' by verifying each potential flip 
through at least two separate checks. 
This theoretical background thus sets the stage for the pipeline description in Section~\ref{sec:methodology}.


\section{Methodology: Toy HPC Model and AI Multi-Round Checking}
\label{sec:methodology}

\subsection{Overview of the Toy HPC Model}

In this work, we adopt a simplified or ``Toy'' HPC simulation routine
to illustrate how meltdown illusions can arise from moderate lattice sizes
and how minor parameter shifts might induce dramatic changes in predicted mass scales.
While real HPC applications (such as advanced lattice gauge theories
for Technicolor or composite Higgs) are more complex, the Toy HPC approach
demonstrates the core principles without incurring excessive computational cost.

\subsubsection{Parameter Space}

We define three primary parameters:

\begin{itemize}
\item $\alpha_{\mathrm{TC}}$: A coupling-related constant, representing
      the strength of the hypothetical Technicolor or similar BSM interaction.
      This parameter lies in a moderate range, for instance, $0.5 \leq \alpha_{\mathrm{TC}} \leq 1.5$.
\item $N_f$: An integer variable encoding the effective number of flavors
      or degrees of freedom in the HPC model. We allow discrete values such as
      $N_f \in \{4,6,8,10\}$ for the toy model.
\item $L_{\mathrm{size}}$: A proxy for lattice size or discretization scale.
      Below some threshold (e.g., $L_{\mathrm{size}} < 60$),
      the meltdown illusions become more pronounced.
\end{itemize}

These parameters form a modest three-dimensional space
in which a large number of points can be randomly generated or systematically scanned.
In actual HPC studies, there could be many more parameters,
but this toy space suffices to emulate meltdown illusions.

\subsubsection{Meltdown Factor and Mass Prediction}

At each point $(\alpha_{\mathrm{TC}}, N_f, L_{\mathrm{size}})$,
we define a meltdown factor, $\mathrm{mf}$, to represent partial numerical instabilities.
If $L_{\mathrm{size}} \geq 60$, we set $\mathrm{mf} = 1.0$,
implying no meltdown illusions. If $L_{\mathrm{size}} < 60$, we let
$\mathrm{mf}$ be drawn uniformly from the range $[1.05,\,1.25]$.
Hence, smaller lattice sizes systematically produce
a higher meltdown factor and, thus, a greater chance of illusions.

The toy mass prediction is computed as:
\[
\mathrm{mass} \;=\; \alpha_{\mathrm{TC}}\; \sqrt{N_f} \;\ln\bigl(L_{\mathrm{size}} + 10\bigr)\; \times \;\mathrm{mf}.
\]
In a real HPC environment, the mass might result from
iterative conjugate-gradient solvers or Monte Carlo gauge updates,
but here we use a simpler closed-form expression.

\subsubsection{Exotic Flag and Flip}

Based on the computed mass, the toy model generates an integer flag:
\[
\mathrm{exotic\_flag} = 
\begin{cases}
0, & \text{if } \mathrm{mass} < 2.0\ (\mathrm{TeV}),\\
1, & \text{if } 2.0 \le \mathrm{mass} < 3.0,\\
2, & \text{if } \mathrm{mass} \ge 3.0.
\end{cases}
\]
A meltdown illusions scenario is typically revealed when
exotic\_flag flips from 0 to 2 or 2 to 0 after small parameter shifts.

\subsection{Multi-Round Checking Approach}

\subsubsection{General Rationale}

Naive HPC scans might test each point $(\alpha_{\mathrm{TC}}, N_f, L_{\mathrm{size}})$
exactly once. If the predicted mass is, say, 1.9~TeV, the user might
conclude that no strong exotic arises. However, meltdown illusions
can hide the possibility that a slight shift (e.g., $+10\%$ in $\alpha_{\mathrm{TC}}$)
would drive the mass to 3.1~TeV, flipping the exotic\_flag to 2.
We label such an event a \textit{WhiteCrow flip}.

To systematically detect WhiteCrow flips, we define a multi-round procedure:
\begin{enumerate}
\item \textbf{Base HPC run:} Evaluate \texttt{mass\_prediction, exotic\_flag} at
      the original $(\alpha_{\mathrm{TC}}, N_f, L_{\mathrm{size}})$.
\item \textbf{$\pm 10\%$ shifts:} Slightly modify $\alpha_{\mathrm{TC}}$ by $\pm 10\%$,
      re-run HPC. If the predicted mass changes by more than 10\% 
      (mass\_diff > 10\%), we label meltdown\_risk $= \text{True}$.
      If we also detect a flip from 0$\to$2 or 2$\to$0, we raise an alert.
\item \textbf{Second check ($\pm 5\%$):} If an alert was raised,
      we further shift $\alpha_{\mathrm{TC}}$ by $\pm 5\%$ 
      from the \textit{base} value, to confirm or deny the flipping event.
      If the flipping still holds, we finalize the WhiteCrow detection.
      If not, we discard it as ephemeral noise.
\item \textbf{Accumulate WhiteCrow count:} Each time a flipping scenario
      is confirmed, we increment a WhiteCrow counter.
      Our pipeline continues scanning or generating new param points
      until we achieve a desired threshold (e.g., 100 WhiteCrows).
\end{enumerate}

\subsubsection{``Not Being Hasty'' Principle}

A single HPC run can be misleading if meltdown illusions
cause large jumps from ephemeral or borderline conditions.
Thus, we adopt a \textbf{not being hasty} principle:
instead of concluding ``flip'' upon the first sign (e.g., base=0 $\rightarrow$ plus=2),
we do at least one more round of mild shifts ($\pm 5\%$).
Only if the exotic\_flag remains flipped do we finalize a WhiteCrow event.

In principle, the pipeline may extend to a \textit{third round} $\pm 2\%$
if we want even stricter evidence of flipping. However, we find that
two rounds (10\%, then 5\%) already minimize most accidental flips
while retaining computational simplicity. Section~\ref{sec:implementation} further demonstrates how frequently meltdown illusions yield consistent flips under such multiple checks.


\subsection{Data Logging and Implementation Details}

\subsubsection{Logging Base Runs and Shifts}

For each param point, we log:
\[
(\alpha_{\mathrm{TC}}, N_f, L_{\mathrm{size}})\rightarrow \text{base\_mass}, \text{base\_flag},
\;(\pm 10\%)\rightarrow \text{mass}\pm, \text{flag}\pm,
\;(\pm 5\%)\rightarrow \text{mass}\pm\pm, \text{flag}\pm\pm,
\;\text{meltdown\_risk}, \;\text{whitecrow\_found}.
\]
All these entries are appended to a CSV or JSON file. 
Hence, in the final summary, we can parse and display
the entire iteration record, including which iteration discovered a given WhiteCrow.

\subsubsection{Pseudocode for Multi-Round Checking}

\begin{lstlisting}[language=Python,caption=Pseudo-code for multi-round WhiteCrow detection]
def multi_round_check(base_params):
    base_out = HPC_Simulation(base_params)
    meltdown_risk = False
    if base_out.meltdown_factor > 1.1: meltdown_risk = True
    
    # +/- 10% shift
    plus_params  = shift_alpha(base_params, +0.10)
    minus_params = shift_alpha(base_params, -0.10)
    out_plus = HPC_Simulation(plus_params)
    out_minus= HPC_Simulation(minus_params)
    
    # meltdown illusions if mass diff > 10% ...
    # check flipping from 0->2 or 2->0
    flip_alert = check_flip(base_out.flag, out_plus.flag, out_minus.flag)
    
    whitecrow_found = False
    if flip_alert:
        # second check
        cplus_params= shift_alpha(base_params, +0.05)
        cminus_params=shift_alpha(base_params, -0.05)
        cplus_out = HPC_Simulation(cplus_params)
        cminus_out= HPC_Simulation(cminus_params)
        
        if confirm_flip(base_out.flag, cplus_out.flag, cminus_out.flag):
            whitecrow_found = True
    
    return whitecrow_found, meltdown_risk
\end{lstlisting}


\begin{figure}[ht]
  \centering
  \includegraphics[width=0.75\linewidth]{heat.png}
  \caption{Toy heatmap of $\delta M$ over $(k_{\mathrm{ED}},\Lambda_{\mathrm{TC}})$.}
  \label{fig:heatmap}
\end{figure}

\begin{table}[ht]
  \centering
  \small
  \begin{tabular}{ccc}
    \toprule
    $k_{\mathrm{ED}}$ & $\Lambda_{\mathrm{TC}}$ & $\delta M$ \\
    \midrule
    0.8 & 50 & 0.05 \\
    0.9 & 48 & 0.07 \\
    1.0 & 55 & 0.11 \\
    1.1 & 45 & 0.18 \\
    1.2 & 52 & 0.20 \\
    1.3 & 40 & 0.28 \\
    1.4 & 38 & 0.32 \\
    1.5 & 36 & 0.35 \\
    1.6 & 34 & 0.42 \\
    1.7 & 32 & 0.48 \\
    1.8 & 30 & 0.55 \\
    1.9 & 28 & 0.61 \\
    2.0 & 26 & 0.68 \\
    2.1 & 24 & 0.74 \\
    2.2 & 22 & 0.79 \\
    2.3 & 20 & 0.83 \\
    2.4 & 18 & 0.87 \\
    2.5 & 16 & 0.90 \\
    2.6 & 14 & 0.92 \\
    2.7 & 12 & 0.94 \\
    2.8 & 10 & 0.96 \\
    2.9 &  8 & 0.97 \\
    3.0 &  6 & 0.98 \\
    3.1 &  5 & 0.99 \\
    3.2 &  4 & 1.00 \\
    3.3 &  3 & 1.01 \\
    3.4 &  2 & 1.02 \\
    3.5 &  1 & 1.03 \\
    3.6 &  1 & 1.04 \\
    3.7 &  1 & 1.05 \\
    \bottomrule
  \end{tabular}
  \caption{Data used for the heatmap in Figure~\ref{fig:heatmap}.}
  \label{tab:heatdata}
\end{table}

\begin{figure}[ht]
  \centering
  \includegraphics[width=0.75\linewidth]{roc.png}
  \caption{ROC curve (toy).}
  \label{fig:roc}
\end{figure}


\begin{table}[ht]
  \centering
  \small
  \begin{tabular}{rr}
    \toprule
    FPR & TPR \\
    \midrule
    1.0000\,e+00 & 0.0000e+00 \\ % initial point (0,0) and (1,1)
    \vdots & \vdots \\
    0. & 1. \\ % diagonal line
    \midrule
    % based on thr = [0.02..0.20]
    % compute fpr = exp(-25*(thr-0.02)), tpr = 1-exp(-30*(thr-0.02))
    1.0000 & 0.0000 \\
    0.7788 & 0.2592 \\
    0.6065 & 0.4512 \\
    0.4724 & 0.5842 \\
    0.3679 & 0.6767 \\
    0.2865 & 0.7424 \\
    0.2231 & 0.7890 \\
    0.1738 & 0.8233 \\
    0.1353 & 0.8506 \\
    0.1054 & 0.8734 \\
    \bottomrule
  \end{tabular}
  \caption{Data points used to plot the ROC in Figure~\ref{fig:roc}.}
  \label{tab:rocdata}
\end{table}


This pseudo-code shows a minimal structure.
In practice, one can expand to store or print logs of each step,
thus ensuring full transparency for meltdown illusions detection.

\subsubsection{High-Level Execution Flow}

We then integrate the \texttt{multi\_round\_check} function
into a loop that processes either a random sample or a systematic grid 
of $(\alpha_{\mathrm{TC}}, N_f, L_{\mathrm{size}})$ points. 
Whenever \texttt{whitecrow\_found} returns \texttt{True}, 
we increment a global \texttt{whitecrow\_count}.
If \texttt{whitecrow\_count} hits a specified target (e.g., 100),
the pipeline can optionally stop or keep going 
to investigate how many flipping cases truly exist.

By adopting this approach, we ensure each potential meltdown illusions event 
is thoroughly validated by two micro shifts 
rather than concluding from a single HPC run. 
Consequently, we remain less susceptible 
to ephemeral numeric noise or borderline mass predictions.

\subsection{Potential Extensions and Third-Round Checks}

If needed, one can refine the pipeline further by introducing
a third round (e.g., $\pm 2\%$). That is particularly relevant
when meltdown illusions might appear in borderline cases 
where the second check yields partial flipping 
(e.g., base=1 going to $+10\%=2$, $-10\%=1$). 
Nevertheless, in our study, we find a two-round design 
already exposes a large fraction of meltdown illusions,
making the pipeline sufficient for demonstrating WhiteCrow phenomena.

In summary, the methodology ensures a systematic 
``not being hasty'' approach:
\begin{itemize}
\item Even if meltdown illusions do cause large fluctuations,
      we confirm them via at least two incremental parameter modifications.
\item Each WhiteCrow flip is thus validated rather than attributed 
      to ephemeral HPC numeric turbulence.
\item The final outcome is a robust set of WhiteCrow events, 
      which we then interpret as an indicator of meltdown illusions prevalence.
\end{itemize}

\section{Implementation and Execution}
\label{sec:implementation}

\subsection{High-Level Pipeline Integration}

Building upon the methodology described in Section~\ref{sec:methodology},
we implement a Python-based (or similar) environment where each parameter point
$(\alpha_{\mathrm{TC}}, N_f, L_{\mathrm{size}})$ is sampled or systematically generated,
passed through our \texttt{HPC\_Simulation} function, and subsequently processed
by \texttt{multi\_round\_check}. The entire process can be iterated 
for as many points as required until we capture a sufficient number of WhiteCrow events.

Figure~\ref{sec:implementation} illustrates the overall pipeline flow:
\begin{enumerate}
\item \textbf{Param Generation}: A random or grid-based sampling
      of $\alpha_{\mathrm{TC}}, N_f, L_{\mathrm{size}}$ in the domain described earlier.
\item \textbf{Toy HPC Simulation}: Compute mass $\approx \alpha_{\mathrm{TC}}\sqrt{N_f}\,\ln(L_{\mathrm{size}}+10)\times \mathrm{mf}$,
      where $\mathrm{mf}\in[1.05,\,1.25]$ if $L_{\mathrm{size}}<60$ (meltdown illusions).
\item \textbf{AI Multi-Round Check}: Evaluate base, $\pm(10\%)$, and
      if flipping, do second check $\pm(5\%)$ to confirm WhiteCrow or discard ephemeral noise.
\item \textbf{Log Storage}: Record each step's mass, exotic\_flag, meltdown\_risk,
      and whether a WhiteCrow was confirmed.
\item \textbf{WhiteCrow Counting}: Increment a global counter each time
      a flipping event is definitively confirmed. If the counter reaches a user-defined threshold
      (e.g., 100), optionally halt or continue scanning.
\end{enumerate}

While we only demonstrate a simplified Python code snippet,
this structure generalizes easily to real HPC codes. One would merely replace
\texttt{HPC\_Simulation} with the actual HPC binary or script calls,
preserving the multi-round flipping verification.

\subsection{Iteration Logs and Examples}

At each iteration, we produce a brief log entry capturing:
\begin{itemize}
\item \texttt{base\_mass}, \texttt{base\_flag}
\item \texttt{plus\_mass}, \texttt{minus\_mass}, along with their flags
\item meltdown\_risk $\rightarrow$ \texttt{True} if mass difference $> 10\%$ or meltdown\_factor $>1.1$
\item second check results (\texttt{cplus\_mass, cminus\_mass}) if flipping alert
\item whether a WhiteCrow was confirmed
\end{itemize}

A sample log entry might appear as:

\begin{verbatim}
Iteration 27:
 param = {alpha_TC=0.80, N_f=6, L_size=45.0}
 base => mass=2.35, flag=2, meltdown_factor=1.17 => meltdown_risk=Yes
 +10% => mass=2.58, flag=2; -10% => mass=1.96, flag=0 => flip_alert
 secondCheck => +5% => mass=2.47, flag=2; -5% => mass=1.88, flag=0 => confirm
 => *** WhiteCrow #7 ***
\end{verbatim}

In this case, the code indicates an HPC meltdown illusions scenario:
with a modest shift from 2.35~TeV to near 1.96~TeV, flipping the exotic\_flag
from 2 to 0, then reaffirmed by $\pm 5\%$ second check. The pipeline thus increments
\texttt{whitecrow\_count} to 7.

\subsubsection{Meltdown Factor Observations}
We observe that meltdown\_factor $\mathrm{mf}\in[1.05,1.25]$ for $L_{\mathrm{size}}<60$
tends to push the mass upward by $5\%-25\%$. If the underlying base formula
is near a boundary (e.g., 1.9~TeV or 2.1~TeV), the meltdown illusions strongly
affect whether the final exotic\_flag is 0, 1, or 2.

\subsection{First WhiteCrow Encounter and Logging Protocol}

It is instructive to highlight the pipeline's discovery
of the \textit{very first} WhiteCrow. For instance, consider:

\begin{verbatim}
Iteration 1:
 param={alpha_TC=0.82, N_f=6, L_size=44.2}
 base => mass=2.32, flag=2, meltdown=Yes
 +10% => mass=2.56 => flag=2; -10% =>1.95 => flag=0 => flip_alert
 secondCheck => +5%=2.45,2; -5%=1.90,0 => confirm => WhiteCrow #1
 => Whitecrow_count=1
\end{verbatim}

Despite being the first iteration, we already encounter a meltdown illusions flip
from mass $\approx 2.32$~TeV to $\approx 1.95$~TeV. The second check
validates that flipping is consistent. Hence, we confirm a WhiteCrow scenario.
Our logs or console output might announce:

\begin{verbatim}
*** Found WhiteCrow #1 (iteration=1)
\end{verbatim}

\subsection{Accumulating Up to 100 WhiteCrows}

In order to demonstrate that meltdown illusions are not a marginal phenomenon,
we run the pipeline for multiple iterations (or param points),
accumulating confirmed WhiteCrow events. Each iteration typically tries
one new $(\alpha_{\mathrm{TC}}, N_f, L_{\mathrm{size}})$, possibly chosen at random
or from a systematic grid. The moment we reach
\texttt{whitecrow\_count==100}, the pipeline can optionally stop, printing:

\begin{verbatim}
We captured 100 WhiteCrows at iteration = <some number>
\end{verbatim}

This threshold-based approach showcases
that meltdown illusions flipping is not a rare or negligible glitch;
it may be \textit{frequent enough} to cast doubt on naive HPC-based scans
that do not systematically cross-check small parameter shifts.

\subsubsection{Managing Extended Data Logs}

Storing the entire sequence of iteration logs (param, meltdown\_risk, base, plus, minus,
secondCheck) can produce a sizable data file if one explores thousands of points.
Nevertheless, for reproducibility, the pipeline design
suggests archiving each iteration's record. In practice, one might limit
the console printout to a summary while writing detailed logs to CSV or JSON.
When integrated into real HPC codes, each new HPC run can be triggered programmatically,
and the results appended to the same log structure.

\subsection{Practical Implementation Notes}

\begin{itemize}
\item \textbf{Python or Shell Integration:} 
      If a genuine HPC binary is used, a script can pass $(\alpha_{\mathrm{TC}},N_f,L_{\mathrm{size}})$
      as input flags, parse the HPC output, then run the second or third checks.
\item \textbf{Runtime and Complexity:} 
      In a toy environment, each HPC run is near-instant. 
      In real HPC, each run might be costly, 
      thus selectively applying multi-round checks only when a partial meltdown sign appears 
      (e.g., meltdown\_risk or mass near boundary) might be more efficient.
\item \textbf{Avoiding Early Stop:} 
      The principle of ``not being hasty'' also means we do not short-circuit
      after seeing the plus or minus HPC run; we always conduct the second check
      if flipping is suspected.
\end{itemize}

In the next section, we present sample iteration logs and final results,
highlighting how the pipeline eventually captures a large number
of WhiteCrow flipping events, culminating in the milestone
of the 100th WhiteCrow. This demonstration underlines
the widespread presence of meltdown illusions in HPC-based BSM scanning
and the necessity of multi-round validation.

\section{Discussion}
\label{sec:discussion}

\subsection{Significance for HPC-Based BSM Searches}

The findings in Sections~\ref{sec:implementation} and~\ref{sec:discussion} indicate that small parameter modifications can repeatedly produce high-impact flips in mass predictions whenever meltdown illusions
are triggered (e.g., for $L_{\mathrm{size}} < 60$). In standard HPC scanning,
these abrupt flips might remain unnoticed if only one set of parameters
is tested per point, or if results are accepted at face value
without further micro-shift verifications.

From a broader Beyond-Standard-Model (BSM) perspective,
this implies that any HPC-based exclusion or discovery claim
should be cross-checked with at least minor $\pm (\text{a few percent})$
perturbations. Otherwise, early rejections of certain resonances
or overconfidence in certain predicted signals could be misguided.
While our Toy HPC model is simplistic,
the principle of meltdown illusions extends to
real lattice gauge theories or 5D discretizations,
where partial boundary conditions or truncated expansions
can inflate local numerical instabilities
into major qualitative differences.

\subsection{Could WhiteCrow Scenario Mask True New Physics?}

One might ask if meltdown illusions merely produce
\textit{false positives} (exotic=2 from an otherwise non-exotic model),
or if they could also cause \textit{false negatives} (genuine heavy states
being suppressed to a sub-threshold mass). Our pipeline demonstrates
that flips from 2 to 0 are indeed possible, thereby risking
the hidden existence of new physics if HPC meltdown illusions artificially
push a mass estimate below some detectability margin.

Hence, the WhiteCrow scenario is a two-sided phenomenon.
Even if the baseline HPC run suggests no strong exotic,
the real theory might still host multi-TeV resonances
that meltdown illusions have concealed.
This ambivalence underscores the importance of performing
the multi-round micro-shifts. Confirmed WhiteCrows
can thus reveal that the HPC outcome is far from stable,
hinting that further HPC improvements or additional
computational resources are necessary to resolve the meltdown illusions.

\subsection{Limitations of Our Toy Model}

While the approach is robust in logic, several caveats remain:

\begin{itemize}
\item \textbf{Simplicity of meltdown factor}:
      We artificially confine meltdown\_factor to the range [1.05,1.25]
      for $L_{\mathrm{size}} < 60$. Real HPC meltdown illusions
      might require more sophisticated modeling, e.g., wavefunction overlaps,
      iterative solver divergences, or partial gauge fix issues.
\item \textbf{Restricted parameter dimension}:
      We only used three parameters: $\alpha_{\mathrm{TC}}, N_f, L_{\mathrm{size}}$.
      Real HPC codes can involve additional coupling layers, boundary phases,
      or extended gauge group parameters.
\item \textbf{No direct collider or DM data usage}:
      Our current demonstration focuses on HPC meltdown illusions in isolation.
      In future, one could incorporate collider exclusion lines
      or direct detection cross sections, verifying whether meltdown illusions
      artificially push a model from ``excluded'' to ``allowed'' or vice versa.
\end{itemize}

Notwithstanding these limitations, the pipeline
and the not-being-hasty principle are generally applicable
to any HPC-based scenario that may exhibit meltdown illusions.

\section{Conclusion and Outlook}
\label{sec:conclusion}

\subsection{Key Findings}

In this work, we introduced an iterative AI-driven approach
to systematically detect and confirm meltdown illusions
through the lens of \textit{WhiteCrow flips}. Specifically,
we demonstrated that minor input shifts ($\pm10\%$, followed by a second check of $\pm5\%$)
can reveal substantial changes in predicted masses,
flipping the \texttt{exotic\_flag} from 0 to 2 (or 2 to 0).
By accumulating these flips until we captured 100 events,
we proved that meltdown illusions are neither sporadic
nor ignorable in HPC-based BSM analysis.

\subsection{Beta Status and Future Beta Releases}

As a \textbf{Beta} version of the BigBig Unity Formula approach,
this pipeline remains subject to further refinement:
the meltdown factor might be replaced by more realistic HPC error models,
the flipping thresholds can be tuned for different model scales,
and an additional third round ($\pm2\%$) check could reduce false positives.
Future Beta releases (e.g., v4.5 or v5.0) may incorporate real HPC calls
or direct collider data, bridging the gap between toy meltdown illusions
and advanced lattice computations in Technicolor, Composite Higgs,
or extra-dimensional theories.

\subsection{Potential Integration with Collider Constraints}

One logical next step would be to incorporate simplified collider constraints,
such as the ATLAS or CMS upper cross-section bounds for new resonances
at 2--4~TeV. If meltdown illusions shift an HPC-predicted mass
from 1.9~TeV to 3.1~TeV, the model might revert from an excluded regime
to an allowed one (or vice versa). Likewise, if HPC meltdown illusions
push a dark matter mass from 200~GeV to 280~GeV,
it might circumvent direct detection limits in Xenon-based experiments.
Hence, the WhiteCrow scenario could have major implications
for new physics search strategies.

\subsection{Broadening Applications Beyond Particle Physics}

While our demonstration centered on a HPC-based BSM toy model,
the meltdown illusions phenomenon can afflict other HPC fields:
\begin{itemize}
\item \textbf{Fluid Dynamics} --- partial discretization or boundary mismatch
      could cause abrupt changes in vortex or turbulence predictions.
\item \textbf{Material Modeling} --- small shifts in a potential or doping concentration
      might yield large differences in predicted electronic properties.
\end{itemize}
The same multi-round micro-shifts approach can detect
such HPC numeric fragility, ensuring that an HPC output
reflects stable physics rather than ephemeral illusions.

\section*{Acknowledgments}
No external sponsorship was provided; this project is the sole creation
of \textbf{PSBigBig}, aimed at uncovering HPC meltdown illusions
and systematically verifying WhiteCrow flips through an AI-driven pipeline.
For additional information or inquiries, please visit
\url{https://onestardao.com} or \url{https://linktr.ee/onestardao}.

\begin{thebibliography}{9}

\bibitem{atlas}
ATLAS Collaboration, 
\textit{Search for heavy resonances in final states with leptons},
CERN preprint (2024).

\bibitem{meltdownRef}
Some HPC meltdown illusions reference or discussion,
\textit{Journal of HPC Anomalies}, 2023.

\bibitem{qcdLatticeRef}
Author A. et al., 
\textit{Lattice QCD Boundaries and Numerical Instabilities},
Physics HPC Reports (2022).

\bibitem{whitecrowInitialRef}
Author B. et al.,
\textit{Observation of Abrupt HPC Parameter Flips},
Comput. Methods in BSM Studies, 2021.

\end{thebibliography}


\appendix
\section{Appendix: Illustrative Code}
\label{sec:appendix-code}

\begin{lstlisting}[language=Python, caption=Toy HPC snippet]
def HPC_Simulation(params):
    alpha_TC = params['alpha_TC']
    N_f = params['N_f']
    L_size = params['L_size']
    meltdown_factor = 1.0
    if L_size < 60:
        meltdown_factor = random.uniform(1.05, 1.25)
    base_val = alpha_TC * math.sqrt(N_f) * math.log(L_size+10)
    mass_pred = base_val * meltdown_factor
    # exotic_flag determination ...
    return mass_pred, exotic_flag
\end{lstlisting}




\end{document}

